\documentclass{article}

\usepackage[margin=1in]{geometry}
\usepackage{minted}
\usepackage{amsmath}
\usepackage{amssymb}
\usepackage{graphicx}
\usepackage{xcolor}
\usepackage{setspace}
\usepackage{hyperref}
\usepackage[backend=biber]{biblatex}

\bibliography{progress-report.bib}
\doublespacing

\title{Final Project Progress Report}
\author{Leanna Calla\\Michael Stergianis}
\date{\today}

\begin{document}
\maketitle
\section{Interpolation}
\label{sec:interpolation}
Much of work in super resolution has to do with interpolation.  There
are many interpolation schemes that can be used in order to increase
the resolution of images.
%
%
\section{Uses}
\label{sec:uses}

Super resolution is incredibly useful. As noted in
\cite{Yang2010ImageSH}, there are four main areas where super
resolution particularly useful.
\subsection{Surveillance Video}
As noted in \cite{Yang2010ImageSH}, super resolution has applications
when a surveillance video is frozen and an specific area is
zoomed. This may help to look at things like faces or license plates.
\subsection{Remote Sensing}

\subsection{Medical Imaging}
Super resolution has applications in Medical Imaging such as improving
the quality of MRI images. \cite{Peled} notes that the quality of the resolution of
an MRI is limited has limitations such as the imaging time and any
movement of the patient. The diffusion-sensitized echo-planar imaging
technique is acquires the image in shot, avoiding phase problems. Some
limitations do remain for brain images.  

\subsection{Video Standard Conversion}
%
Low resolution images can be combined into one higher resolution
image. This is known as multiple image super resolution.
%
%
\section{Necessity}
\label{sec:necessity}
%
%
\section{Future Work}
\label{sec:future}
Hello world
%
% Bibliography
\printbibliography
\end{document}
