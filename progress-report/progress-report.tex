\documentclass{article}

\usepackage[margin=1in]{geometry}
\usepackage{minted}
\usepackage{amsmath}
\usepackage{amssymb}
\usepackage{graphicx}
\usepackage{xcolor}
\usepackage{setspace}
\usepackage{hyperref}
\usepackage[backend=biber]{biblatex}

\bibliography{progress-report.bib}
\doublespacing

\title{Final Project Progress Report}
\author{Leanna Calla\\Michael Stergianis}
\date{\today}

\begin{document}
\maketitle
\section{Interpolation}
\label{sec:interpolation}
Much of work in super resolution has to do with interpolation.  There
are many interpolation schemes that can be used in order to increase
the resolution of images. This section will briefly describe a few
interpolation schemes of note.
%
\subsection{Bilinear Interpolation}
\label{subsec:bilinear}
Bilinear interpolation is the process of supposing a subpixel
surrounded by four real pixels and interpolating linearly between the
four pixels to find the subpixel.
%
%
\section{Uses}
\label{sec:uses}
Super resolution is incredibly useful. In other works \cite{Yang2010ImageSH} reasons you
would want to use super resolution are as follows. Some areas where
Super 
%
Low resolution images can be combined into one higher resolution
image. This is known as multiple image super resolution.
%
%
\section{Necessity}
\label{sec:necessity}
%
%
\section{Future Work}
\label{sec:future}
Hello world
%
% Bibliography
\printbibliography
\end{document}
