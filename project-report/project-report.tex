\documentclass{article}

\usepackage[margin = 1in]{geometry}
\usepackage{minted}
\usepackage{amsmath}
\usepackage{amssymb}
\usepackage{graphicx}
\usepackage{subcaption}
\usepackage{xcolor}
\usepackage{setspace}
\usepackage{hyperref}
\usepackage[backend=biber]{biblatex}

\bibliography{project-report.bib}
\doublespacing

\title{Final Project Report}
\author{Leanna Calla \\ Michael Stergianis}
\date{March 27th, 2018}

\begin{document}
\maketitle
%
\section{Introduction}
\begin{flushleft}
  %
  When attempting to increase the detail of an image, any noise in
  that image will also be used in the interpolation scheme. Therefore
  an integral first step to super resolution is to denoise the image
  in a way that preserves edges, corners, and the finer details of the
  image.
  %
\end{flushleft}
%
\section{Filtering Techniques}
\subsection{Bilateral Filtering}
\subsection{Median Filtering}
Super resolution techniques often requires some de-blurring
techniques. Median filtering is a common technique to remove noise
from an image. 
%
\end{document}
