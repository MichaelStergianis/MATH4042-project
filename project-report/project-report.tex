\documentclass{article}

\usepackage[margin = 1in]{geometry}
\usepackage{minted}
\usepackage{amsmath}
\usepackage{amssymb}
\usepackage{graphicx}
\usepackage{subcaption}
\usepackage{xcolor}
\usepackage{setspace}
\usepackage{hyperref}
\usepackage[backend=biber]{biblatex}

\bibliography{project-report.bib}
\doublespacing

\title{Final Project Report}
\author{Leanna Calla \\ Michael Stergianis}
\date{March 27th, 2018}

\begin{document}
\maketitle
%
\section{Introduction}
\begin{flushleft}
  %
  When attempting to increase the detail of an image, any noise in
  that image will also be used in the interpolation scheme. Therefore
  an integral first step to super resolution is to denoise the image
  in a way that preserves edges, corners, and the finer details of the
  image.
  %
\end{flushleft}
%
\section{Filtering Techniques}
\subsection{Bilateral Filtering}
Bilateral filtering is useful in smoothing images while also
preserving edges \cite{Paris}.
\subsection{Gaussian Filtering}
\subsection{Median Filtering}
%
\begin{flushleft}
  %
  Median filtering is a common technique to remove noise from an
  image. \cite{Med2012} explores the median filter. A median filter is a
  nonlinear filter. A sliding mask is applied to the image, where the
  value of a noisy pixel is replaced by the median of the pixels in
  the mask. The performance of the median filter depends on the size
  of the mask and the distribution of the noise.\\
  %
   \cite{improved-median} and \cite{Med2012} shows progress on an
   improved median filter over a year. The algorithm in
   \texttt{improved\_median} has been implemented to be compared with
   the standard median filter. The steps of the algorithm are as
   follows
   \begin{enumerate}
   \item an $n\times n$ mask is chosen to slide over the image
     \item the median, \textit{median}, of the mask is computed. This
       is done by ordering the elements and then choosing the middle
       one.
       \item a new average is computed, \textit{avg}. In this average the
         central element is replaced with \textit{median}.
         \item Each pixel in the mask is compared with
           \textit{avg}
           \item if every pixel in the mask is greater than the
             \textit{avg}, then the central pixel is replaced with the
             median of the patch
           \item otherwise the patch is unchanged
             \item steps 4-6 will be repeated until the image has been
               completely traversed.
     \end{enumerate}
   
  \end{flushleft}
%
\section{Interpolation Techniques}
%
\section{Statistics Tracking}
%
% Bibliography
\newpage
\printbibliography
\end{document}
