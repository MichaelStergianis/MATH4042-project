\documentclass{article}

\usepackage[margin = 1in]{geometry}
\usepackage{minted}
\usepackage{amsmath}
\usepackage{amssymb}
\usepackage{graphicx}
\usepackage{subcaption}
\usepackage{xcolor}
\usepackage{setspace}
\usepackage{hyperref}
\usepackage[backend=biber]{biblatex}

\bibliography{project-report.bib}
\doublespacing

\title{Final Project Report}
\author{Leanna Calla \\ Michael Stergianis}
\date{March 27th, 2018}

\begin{document}
\maketitle

\section{Introduction}

\section{Filtering Techniques}
\subsection{Bilateral Filtering}
Bilateral filtering is useful in smoothing images while also
preserving edges \cite{Paris}.
\subsection{Gaussian Filtering}
\subsection{Median Filtering}
 Median filtering is a common technique to remove noise
from an image. \cite{Med} explores the median filter. A median filter is a nonlinear filter. A sliding mask
is applied to the image, where the value of a noisy pixel is replaced
by the median of the pixels in the mask. The performance of the median
filter depends on the size of the mask and the distribution of the
noise.
\section{Interpolation Techniques}

\section{Statistics Tracking}
\newpage
% Bibliography
\printbibliography


\end{document}
